\documentclass{report}
\usepackage{fullpage}
\usepackage{parskip}
\pagestyle{empty}
\begin{document}
\begin{center}
\begin{Huge}
\textbf{Project Proposal}
\end{Huge}

Matt Kline, James Gordon, Tyler Kuske, and Creighton Long
\end{center}

\section*{Overview}
The proposed project is a virtual shooting range.
The setup would consist of multiple targets which could be placed around a room.
Each target would consist of an IR receiver as well as some LEDs, and would be connected to our ZedBoard.

Multiple players would then have some sort of ``gun'' (perhaps a modified airsoft gun or a 3D-printed frame)
consisting of a processor package, an IR laser, a speaker, and a wireless transmitter.
Each gun would be configured to emit a unique IR sequence when the trigger is pulled,
which the ZedBoard/target setup would use to determine which player hit a target.
Further work is mostly software-based and several games and scenarios,
such as timing drills or hitting multiple targets in a sequence, could be played with one or more players.
Statistics will be tracked by the ZedBoard and can be displayed using a desktop or mobile application.

\section*{Primary Goals}

The following goals should be considered basic benchmarks for the project.

\begin{itemize}
\item At least two IR ``guns'', each able to emit their own unique IR signature, sound, and wireless signal
\item At least three target modules with IR receivers and LEDs
\item At least one single player game/drill
\item At least one multiplayer game/drill
\item Statistic tracking via the ZedBoard
\item Statistic display and game controls via a desktop or mobile application
\end{itemize}

\section*{Secondary Goals}

The following goals are additional features to be developed once the primary goals are met.

\begin{itemize}
\item Additional targets
\item Pop-up or moving targets
\item Force feedback on the ``guns'' to simulate recoil
\item Additional game modes and statistics
\item Ability to track stability while aiming (likely through the use of an accelerometer)
\end{itemize}

\end{document}
